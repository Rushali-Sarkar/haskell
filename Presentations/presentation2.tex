\documentclass[10pt]{beamer}
\usepackage[utf8x]{inputenc}
\usepackage{hyperref}
\usepackage{fontawesome}
\usepackage{graphicx}
\usepackage[english,ngerman]{babel}

% ------------------------------------------------------------------------------
% Use the beautiful metropolis beamer template
% ------------------------------------------------------------------------------
\usepackage[T1]{fontenc}
\usepackage{fontawesome}
\usepackage{FiraSans} 
\mode<presentation>
{
  \usetheme[progressbar=foot,numbering=fraction,background=light]{metropolis} 
  \usecolortheme{default} % or try albatross, beaver, crane, ...
  \usefonttheme{default}  % or try serif, structurebold, ...
  \setbeamertemplate{navigation symbols}{}
  \setbeamertemplate{caption}[numbered]
  %\setbeamertemplate{frame footer}{My custom footer}
} 

% ------------------------------------------------------------------------------
% beamer doesn't have texttt defined, but I usually want it anyway
% ------------------------------------------------------------------------------
\let\textttorig\texttt
\renewcommand<>{\texttt}[1]{%
  \only#2{\textttorig{#1}}%
}

% ------------------------------------------------------------------------------
% minted
% ------------------------------------------------------------------------------
\usepackage{minted}


% ------------------------------------------------------------------------------
% tcolorbox / tcblisting
% ------------------------------------------------------------------------------
\usepackage{xcolor}
\definecolor{codecolor}{HTML}{FFC300}

\usepackage{tcolorbox}
\tcbuselibrary{most,listingsutf8,minted}

\tcbset{tcbox width=auto,left=1mm,top=1mm,bottom=1mm,
right=1mm,boxsep=1mm,middle=1pt}

\newtcblisting{myr}[1]{colback=codecolor!5,colframe=codecolor!80!black,listing only, 
minted options={numbers=left, style=tcblatex,fontsize=\tiny,breaklines,autogobble,linenos,numbersep=3mm},
left=5mm,enhanced,
title=#1, fonttitle=\bfseries,
listing engine=minted,minted language=r}


% ------------------------------------------------------------------------------
% Listings
% ------------------------------------------------------------------------------
\definecolor{mygreen}{HTML}{37980D}
\definecolor{myblue}{HTML}{0D089F}
\definecolor{myred}{HTML}{98290D}

\usepackage{listings}

% the following is optional to configure custom highlighting
\lstdefinelanguage{XML}
{
  morestring=[b]",
  morecomment=[s]{<!--}{-->},
  morestring=[s]{>}{<},
  morekeywords={ref,xmlns,version,type,canonicalRef,metr,real,target}% list your attributes here
}

\lstdefinestyle{myxml}{
language=XML,
showspaces=false,
showtabs=false,
basicstyle=\ttfamily,
columns=fullflexible,
breaklines=true,
showstringspaces=false,
breakatwhitespace=true,
escapeinside={(*@}{@*)},
basicstyle=\color{mygreen}\ttfamily,%\footnotesize,
stringstyle=\color{myred},
commentstyle=\color{myblue}\upshape,
keywordstyle=\color{myblue}\bfseries,
}


% ------------------------------------------------------------------------------
% The Document
% ------------------------------------------------------------------------------
\title{Monads and Typeclasses}
\subtitle{Inspired by the book \\``Finding Success and Failure in Haskell" \\by Julie Moronuki and Chris Martin}
\author{Rushali Sarkar}
\date{August 2021}

\begin{document}

\maketitle

\section{Introduction}
\begin{frame}{Introduction}
Monad is a typeclass and a typeclass defines a set of generic functions that work with a set of types.\\
\texttt{\vspace{1cm}\\
Prelude> :type (+)\\
(+) :: Num a => a -> a -> a}
\end{frame}


\section{Why Monads?}
\begin{frame}{Why Monads ?}
Monad encapsulates the idea of merging two things together into a single one, with the possibility that one of those things is irrelevant.\\
\begin{center}
\vspace{0.1cm}
\color{blue}
\textbf{Inspired from the Article by Elviro Rocca}\\
\href{https://broomburgo.github.io/fun-ios/post/why-monads/}{https://broomburgo.github.io/fun-ios/post/why-monads/}
\end{center}
\end{frame}

\section{Let's Get Coding}

\begin{frame}[fragile]{Code}
\begin{myr}{Code with Error}
errorHead :: [a] -> a 
errorHead (x: xs) = x
\end{myr}
\begin{myr}{A Possible Fix}
safeHead :: [a] -> Maybe a
safeHead [] = Nothing
safeHead (x: xs) = Just x
\end{myr}
\end{frame}

\section{Another Example}

\begin{frame}[fragile]{Validate Password}
\begin{lstlisting}[language=haskell]
validatePassword :: String -> Maybe String
validatePassword password = 
  case (cleanWhiteSpace password) of
    Nothing -> Nothing
    Just password2 -> 
      case (requireAlphaNum password2) of 
        Nothing -> Nothing
        Just password3 -> 
          case (checkPasswordLength password3) of
            Nothing -> Nothing
            Just password4 -> password4
\end{lstlisting}
\end{frame}

\begin{frame}[fragile]{Another Example}
\begin{myr}{Sample Code}
convertString x = map toUpper (reverse x)
\end{myr}
\begin{myr}{Using Infix Function Composition Operator}
convertString = map toUpper.reverse
\end{myr}
\end{frame}

\begin{frame}[fragile]{Bind Operator}
\Large{
\textbf{
Prelude> :type (>>=)\\
(>>=) :: Monad m => m a -> (a -> m b) -> m b}}
\end{frame}

\begin{frame}{Monads making life Easy}
validatePassword :: String -> Maybe String\\
validatePassword password = 
  cleanWhiteSpace password >>= requireAlphaNum >>= checkPasswordLength\\
\end{frame}


\begin{frame}[standout]
    Thank You Everyone ~\alert{\faSmileO}~
\end{frame}

\end{document}
